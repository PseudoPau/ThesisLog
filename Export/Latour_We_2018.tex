% Options for packages loaded elsewhere
\PassOptionsToPackage{unicode}{hyperref}
\PassOptionsToPackage{hyphens}{url}
%
\documentclass[
]{article}
\usepackage{amsmath,amssymb}
\usepackage{lmodern}
\usepackage{iftex}
\ifPDFTeX
  \usepackage[T1]{fontenc}
  \usepackage[utf8]{inputenc}
  \usepackage{textcomp} % provide euro and other symbols
\else % if luatex or xetex
  \usepackage{unicode-math}
  \defaultfontfeatures{Scale=MatchLowercase}
  \defaultfontfeatures[\rmfamily]{Ligatures=TeX,Scale=1}
\fi
% Use upquote if available, for straight quotes in verbatim environments
\IfFileExists{upquote.sty}{\usepackage{upquote}}{}
\IfFileExists{microtype.sty}{% use microtype if available
  \usepackage[]{microtype}
  \UseMicrotypeSet[protrusion]{basicmath} % disable protrusion for tt fonts
}{}
\makeatletter
\@ifundefined{KOMAClassName}{% if non-KOMA class
  \IfFileExists{parskip.sty}{%
    \usepackage{parskip}
  }{% else
    \setlength{\parindent}{0pt}
    \setlength{\parskip}{6pt plus 2pt minus 1pt}}
}{% if KOMA class
  \KOMAoptions{parskip=half}}
\makeatother
\usepackage{xcolor}
\usepackage{graphicx}
\makeatletter
\def\maxwidth{\ifdim\Gin@nat@width>\linewidth\linewidth\else\Gin@nat@width\fi}
\def\maxheight{\ifdim\Gin@nat@height>\textheight\textheight\else\Gin@nat@height\fi}
\makeatother
% Scale images if necessary, so that they will not overflow the page
% margins by default, and it is still possible to overwrite the defaults
% using explicit options in \includegraphics[width, height, ...]{}
\setkeys{Gin}{width=\maxwidth,height=\maxheight,keepaspectratio}
% Set default figure placement to htbp
\makeatletter
\def\fps@figure{htbp}
\makeatother
\setlength{\emergencystretch}{3em} % prevent overfull lines
\providecommand{\tightlist}{%
  \setlength{\itemsep}{0pt}\setlength{\parskip}{0pt}}
\setcounter{secnumdepth}{-\maxdimen} % remove section numbering
\ifLuaTeX
  \usepackage{selnolig}  % disable illegal ligatures
\fi
\IfFileExists{bookmark.sty}{\usepackage{bookmark}}{\usepackage{hyperref}}
\IfFileExists{xurl.sty}{\usepackage{xurl}}{} % add URL line breaks if available
\urlstyle{same} % disable monospaced font for URLs
\hypersetup{
  pdftitle={Latour\_We\_2018},
  hidelinks,
  pdfcreator={LaTeX via pandoc}}

\title{Latour\_We\_2018}
\author{}
\date{}

\begin{document}
\maketitle

\href{/Users/zhangwei/Documents/ThesisLog/Reading\%20notes/index.html\#readingss}{\#readingss}

\includegraphics{https://d3i71xaburhd42.cloudfront.net/ef3b53d60f8350f7824c614e3cc4179e90f4a8e5/1-FigureBL1-1.png}

\texttt{Spatial\ configuration\ of\ the\ seven\ imaginary\ planets\ proposed\ by\ Bruno\ Latour,\ drawing\ by\ Alexandra\ Arénes,\ 2018}

\begin{center}\rule{0.5\linewidth}{0.5pt}\end{center}

Latour, B. (2018).~\emph{``We Don't Seem to Live on the Same Planet'' --
A Fictional Planetarium.}

\href{http://www.bruno-latour.fr/sites/default/files/162-SEVEN-PLANETS-CZpdf.pdf}{pdf}

\href{/Users/zhangwei/Documents/ThesisLog/Reading\%20notes/index.html\#Bruno_Latour}{\#Bruno\_Latour}
\href{/Users/zhangwei/Documents/ThesisLog/Reading\%20notes/index.html\#Architecture}{\#Architecture}
\href{/Users/zhangwei/Documents/ThesisLog/Reading\%20notes/index.html\#Theory}{\#Theory}

\begin{center}\rule{0.5\linewidth}{0.5pt}\end{center}

\texttt{“Today,\ geopolitics\ is\ also\ concerned\ with\ wars\ over\ the\ definition\ of\ the\ stage\ itself.\ A\ conflict\ will\ be\ called,\ from\ now\ on,\ “of\ planetary\ relevance”\ not\ because\ it\ has\ the\ planet\ for\ a\ stage,\ but\ because\ it\ is\ about\ which\ planet\ you\ are\ claiming\ to\ inhabit\ and\ defend.”}
\emph{(Latour, 2018, p. 1)}

~~~It is important to frame the field and clarify the definition before
we start a conversation. It is crucial to read the information behind
the written words when given a commission from a client to avoid
trouble.

\texttt{“The\ climate\ question\ is\ not\ one\ aspect\ of\ politics\ among\ others,\ but\ that\ which\ defines\ the\ political\ order\ from\ beginning\ to\ end,\ forcing\ all\ of\ us\ to\ redefine\ the\ older\ questions\ of\ social\ justice\ along\ with\ those\ of\ identity,\ subsistence,\ and\ attachment\ to\ place.”}
\emph{(Latour, 2018, p. 1)}

~~~However, politics, economics, ideology, law and some other so-called
``human civilization'' become reasons to prevent human uniting together.
Is everyone really equal speaking of Climate?

\texttt{“Climate\ mutation\ means\ that\ the\ question\ of\ the\ land\ on\ which\ we\ all\ stand\ has\ come\ back\ into\ focus,\ hence\ the\ general\ political\ disorientation,\ especially\ for\ the\ left,\ which\ did\ not\ expect\ to\ have\ to\ talk\ again\ of\ “people”\ and\ “soil”\ —\ questions\ mostly\ abandoned\ to\ the\ right.”}
\emph{(Latour, 2018, p. 1)}

~~~It is not climate that splits us. Climate makes the separation in
groups obvious. Although we are talking about the same thing, different
phrasing can lead to different results. The antagonism floating on the
top makes us too tired to discuss the real important things. The problem
within the contemporary society is that people refuse to join the
discourse.

\texttt{“The\ principle\ that\ will\ lead\ me\ in\ this\ reckoning\ is\ the\ link\ between\ the\ territory\ necessary\ for\ our\ subsistence\ and\ the\ territory\ that\ we\ recognize\ —\ legally,\ affectively\ —\ as\ our\ own\ and\ thus\ as\ the\ source\ of\ our\ freedom\ and\ autonomy.”}
\emph{(Latour, 2018, p. 1)}

~~~Latour mentions two territories, both of which are inhabitable. One
is the productive space that keeps us alive. The other is the container
for our daily activities in the form of data, part of which lies in the
Google\textquotesingle s or Meta\textquotesingle s giant data center.
The non-material world does not exist without a physical carrier. The
first territory is the place we stand, we see and we touch. The second
one is much more complicated because it was broken into pieces of puzzle
spreading everywhere.\\
If we collect the pieces of puzzle in the same physical site, and we try
to assemble them, we will get a beast which can represent our world but
a distorted one. It sounds ridiculous, like finding a chipped fingernail
and saying I found someone hiding here. But the reality is that we can
indeed find the person through bioinformatics...\\
The digital footprints are even harder to eliminate. We have to pray
that no one is stalking us and to pretend we are free people.

\texttt{“the\ assumption\ that\ the\ present\ disorientation\ is\ due\ to\ the\ fabulous\ increase\ in\ the\ lack\ of\ fit\ between\ the\ two\ sets\ of\ constraints:\ we\ inhabit\ as\ citizens\ a\ land\ that\ is\ not\ the\ one\ we\ could\ subsist\ on,\ hence\ the\ increased\ feeling\ of\ homelessness,\ a\ feeling\ that\ is\ transforming\ the\ former\ ecological\ questions\ into\ a\ new\ set\ of\ more\ urgent\ and\ more\ tragic\ political\ struggles.”}
\emph{(Latour, 2018, p. 1)}

~~~Latour brings our attentions to the concept of "citizenship" and
politics. A lot of people are pessimistic about the contemporary society
because almost all non-binary problems will eventually turned into a
politic issues. It is always the position instead of the rationality
that gives the answer, even in an environment where everyone can use a
pseudonym. Our position is limited by our vision. It was determined by
origin, identity, education and environment.\\
Now let\textquotesingle s put these advanced topics aside and talk a
little bit about how to survive the brutal life adventure. Obviously, we
have to equipped with different philosophies and methods to survive in
the two different lands. We should realize the emotion and knowledge are
like the string connecting the two doppelgangers. The lifespan and the
spirit are shared.\\
Different accounts means different doppelgangers. The doppelgangers can
represent a history and may have no relation to the present. I noticed
the music recommendation by my 18\textquotesingle s account was very
different from my 20\textquotesingle s account. Algorithms are defeated
by fickle humans.\\
Speaking of the internet world, we believe the internet friends can
replace the friends in the real life to some extent. Some of us put more
effort into their accounts than tidying their rooms. It has no problem.
I suddenly think of years before people are eager to see or date their
internet friends in person. What kind of space do people need now? Shall
we listen to our desire when we are designing the space? Who are we
designing for?

\texttt{“This\ is\ what\ Pierre\ Charbonnier\ calls\ the\ “ubiquity\ of\ the\ moderns”\ to\ underline\ that\ there\ is\ no\ correspondence\ whatsoever\ between\ the\ shape\ of\ nation\ states\ in\ the\ legal\ sense\ and\ the\ widely\ distributed\ sources\ of\ the\ wealth\ its\ citizens\ benefit\ from.”}
\emph{(Latour, 2018, p. 1)}

~~~It is irrelevant to the context. If being born in the United States
means you are an American, can my digital doppelganger be an American?
By the way, it was born in Google\textquotesingle s servers in 2015.\\
The world is constructed according to the algorithm proposed by
modernism. When the world building starts to go wrong, we are patching
to correct off-tracks. I wonder is it still the same thing with too many
patches?

\texttt{“GLOBALIZATION\ is\ simultaneously\ that\ toward\ which\ the\ whole\ world\ is\ supposed\ to\ have\ progressed\ and\ a\ totally\ skewed\ utopian\ domain\ where\ time\ and\ space\ have\ been\ colonized\ to\ the\ point\ of\ rendering\ it\ uninhabitable\ and\ paralyzing\ any\ reaction\ to\ the\ threat\ everyone\ clearly\ sees\ coming.”}
\emph{(Latour, 2018, p. 2)}

~~~Globalization always renders an optimistic mood. We see the world
through distorted lens. We empathize with capitalists as if we were
rich, too. Like I sometimes cannot help teaching Americans how to spell
English worlds...\\
We are so arrogant that we ignore the crisis with peace of mind. Even we
want to react, we are held down by those around us, families, friends,
etc.

\texttt{“It\ is\ different\ from\ the\ former\ precisely\ because\ it\ began\ to\ rematerialize\ all\ the\ elements\ that\ had\ been\ left\ aside,\ a\ bit\ too\ quickly,\ by\ those\ who\ had\ embarked\ on\ the\ great\ progressive\ movement\ toward\ globalization.”}
\emph{(Latour, 2018, p. 2)}

~~~Anthropocene is an addendum to Globalization. They are in
figure-ground relation. When we discuss Anthropocene, we must be aware
that we are also talking about Globalization. They are shadows to each
others.

\texttt{“The\ key\ point\ is\ that\ it\ is\ not\ nature\ as\ such,\ whose\ immensity,\ indifference,\ aloofness,\ importance,\ and\ all-encompassing\ substance\ have\ always\ been\ celebrated,\ but\ an\ agent\ with\ its\ own\ force\ and\ power\ that\ requests\ to\ be\ integrated,\ in\ some\ way,\ into\ the\ political\ domain.”}
\emph{(Latour, 2018, p. 2)}

I don\textquotesingle t know if it is reasonable to discuss human beings
as if we are not human beings. Latour\textquotesingle s proposal puts
Anthropocene to the same level as individuals.

\texttt{“freedom\ is\ for\ the\ few,\ not\ for\ the\ many;\ breaking\ from\ the\ limits\ of\ nature\ is\ the\ essential\ destiny\ of\ those\ few\ only”}
\emph{(Latour, 2018, p. 3)}

~~~Everybody dreams to be the few.

~~~As architects we should be aware that we are the accomplices of
capital, though we sometimes claim that we are environmental agents.

\texttt{“On\ planet\ EXIT,\ the\ plan\ is\ that\ it\ will\ soon\ be\ possible\ to\ download\ our\ mortal\ bodies\ into\ a\ mix\ of\ robots,\ DNA,\ clouds,\ and\ AI,\ thereby\ situated\ as\ far\ as\ possible\ from\ the\ humble\ and\ limited\ Earth.”}
\emph{(Latour, 2018, p. 3)}

~~~If human beings no longer exist in human form, do we still need to
treat them as the same species? Can Exit solve the problems we are
facing? It reminds me of the Peasant Revolts that they kill their king
and become the new king. The history will repeat itself if we keep a
blind eye to the issues. Exit will make things even more difficult.

\texttt{“Wherever\ the\ gated\ community\ ends\ up\ being\ situated,\ the\ great\ difference\ between\ the\ planets\ GLOBALIZATION\ and\ EXIT\ is\ that\ there\ is\ no\ longer\ any\ project\ for\ the\ billions\ of\ humans\ who\ are\ explicitly\ now\ left\ behind\ or,\ to\ use\ a\ cruel\ but\ frank\ adjective,\ have\ become\ supernumerary.Civilization,in\ the\ narrow\ sense\ of\ a\ project\ invented\ in\ the\ eighteenth\ century,\ is\ now\ abandoned\ for\ good.”}
\emph{(Latour, 2018, p. 3)}

~~~Choosing Planet EXIT was a big gamble, betting that you would be one
of the chosen few. Run-away cannot solve anything but makes things
bigger. When the issue is as big as a planet, we have nowhere to hide.
At that time, arguing who is the best U.S president becomes a funny
scene.

\texttt{“Where\ do\ the\ millions\ of\ people\ go?\ In\ one\ direction\ and\ one\ only:\ wherever\ they\ would\ like,\ so\ long\ as\ they\ remain\ behind\ walls,\ and\ thereby\ retain\ at\ least\ one\ element\ of\ the\ former\ civilizing\ project\ protection\ and\ identity.”}
\emph{(Latour, 2018, p. 3)}

~~~Walls provide sense of solidity. Gravity provides sense of reality.
Those who hide behind walls are not necessarily cowards. They are
rational and a little conservative and believe they can find an image
from the history to deal with the dilemma right now. For some reasons,
they don\textquotesingle t trust themselves. They need tons of reference
to support their courage. They are the majority, the indigenous people
of globalization.

\texttt{“If\ prosperity\ and\ freedom\ are\ gone\ and\ it\ is\ impossible,\ as\ scientists\ insist,\ to\ bring\ prosperity\ and\ earthly\ conditions\ together,\ then\ let’s\ at\ least\ have\ an\ identity,\ a\ sense\ of\ belonging.\ Does\ it\ solve\ the\ problem\ of\ the\ superposition\ between\ subsistence,\ territory,\ and\ freedom?”}
\emph{(Latour, 2018, p. 3)}

~~~What is the dream habitat? Is there a place satisfied "subsistence,
territory, and freedom" needs for its inhabitants?\\
We voluntarily build walls to resist external risks. When we ourselves
become a risk, we passively and voluntarily isolate ourselves. There
exists an invisible wall of consciousness embedded in the culture, or
even in the DNA. Human beings may born to divide themselves.

\texttt{“let’s\ say\ the\ period\ from\ the\ sixteenth\ through\ eighteenth\ century,\ with\ its\ coal,\ and\ the\ nineteenth\ and\ twentieth\ centuries,\ with\ coal\ and\ oil.Economics\ was\ still\ an\ art\ of\ dealing\ with\ prudence\ and\ limits,and\ not\ yet\ with\ what\ could\ render\ as\ invisible\ the\ conditions\ of\ subsistence\ and\ infinitize\ profit.\ To\ use\ Timothy\ Mitchell’s\ thesis,\ they\ could\ not\ render\ invisible\ all\ links\ to\ earthly\ conditions.”}
\emph{(Latour, 2018, p. 4)}

~~~Noise, dirts and bad elements are also parts of the world. The
imperfect is the real.

\texttt{“You\ cannot\ insert\ into\ politics\ just\ any\ sort\ of\ natural\ entity\ without\ transforming\ the\ search\ for\ freedom\ and\ autonomy\ into\ the\ simple\ domination\ of\ necessity\ and\ heteronomy.”}
\emph{(Latour, 2018, p. 6)}

\texttt{“the\ scales\ are\ correct,\ the\ influence\ indisputable,\ the\ effects\ devastating17\ —\ is\ not\ something\ that\ any\ political\ agent\ can\ hear\ without\ ceasing\ to\ be\ a\ human\ political\ agent.\ In\ becoming\ geology,\ anthropocentric\ humans\ have\ become\ as\ immobile\ as\ pillars\ of\ salt.”}
\emph{(Latour, 2018, p. 6)}

~~~Anthropocene is kind of static because it only describes "here" and
"now" and has no relation to past and future. That is its limitation.
But still a good model to perceive the situation.

\texttt{“The\ crucial\ discovery\ is\ that\ life\ forms\ don’t\ reside\ in\ space\ and\ time,\ but\ that\ time\ and\ space\ are\ the\ result\ of\ their\ own\ entanglement.”}
\emph{(Latour, 2018, p. 6)}

~~~Yes. Time and space are concepts defined by humans.There used to be
no digits in the world. When humans abstract the world into the digits,
they possess a new world.

\texttt{“Whenever\ people\ talk\ about\ modernization,\ they\ immediately\ create,\ by\ way\ of\ contrast,\ a\ primeval\ site,\ that\ of\ archaic\ attachment\ to\ the\ soil,\ to\ the\ ground,\ which\ is\ then\ either\ ridiculed\ as\ that\ from\ which\ the\ whole\ civilizing\ project\ has\ been\ extricating\ itself,\ or\ —\ what\ is\ even\ worse\ —\ celebrated\ as\ a\ mythical,\ archaic,\ primordial,\ autochthonous\ Ur-Earth\ free\ from\ all\ the\ tragic\ sins\ of\ civilized\ humans.”}
\emph{(Latour, 2018, p. 6)}

~~~Modernism lacks a unified definition and description, which has led
to future generations guessing to understand it, and therefore prone to
over-interpret or mis-interpret it. When we talk about modernism, what
are we talking about? Modernism is not only a style...

\includegraphics{/Users/zhangwei/Documents/ThesisLog/index.html}

\end{document}
